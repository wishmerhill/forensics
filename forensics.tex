\documentclass[13pt,a4paper]{article}
\usepackage[italian]{babel}
\usepackage[latin1]{inputenc}
\usepackage{amsmath}
\usepackage{amsfonts}
\usepackage{amssymb}
\usepackage{graphicx}
\usepackage{blindtext}
\usepackage{scrextend}
\usepackage[big,rilegatura=.6cm]{layaureo}
\begin{document}
	\tableofcontents
	\section{Digital Forensics}
		\subsection{Premessa}
			Nell'ambito dell'attivit� quotidiana delle forze di polizia � sempre pi� frequente, ormai, imbattersi nella necessit� di conservare, analizzare e valutare materiale cosiddetto informatico o ad alta tecnologia. 
			Gi� la ``semplice'' analisi di un tabulato di traffico telefonico (fonia o eventualmente anche dati) � di fatto una analisi forense, sebbene certamente non sofisticata quanto quella di un computer.
			
			Di fatto, oggi non esiste fenomeno giuridicamente rilevante che non abbia una componente digitale dalla quale estrarre informazioni utili alla ricostruzione dei fatti.
			
			Nel tempo, le procedure operative hanno subito notevoli cambiamenti, dovuti principalmente alle innovazioni intervenute sui mezzi di comunicazione informatica e sulle piattaforme sociali diffusesi negli ultimi tempi.
			Se ad esempio fino a qualche anno fa la regola nel repertare un personal computer acceso era quella di spegnerlo senza particolari cautele (staccando molto semplicemente la spina) per poi procedere alla sigillatura delle porte e alla chiusura del plico.
			Oggi non si opera pi� in questo modo, almeno non sempre, in quanto si � visto che il contenuto della memoria volatile del computer pu� essere determinante o comunque di notevole importanza per le indagini. Come operare allora?
			Non esiste una regola unica e valida per tutte le occasioni, se non quella di applicare in ogni caso il buon senso, nel rispetto del Codice di Procedura Penale, e la sensibilit� investigativa maturata nel corso di anni di indagini, anche tradizionali.
			In quest'ottica, nessun tecnico puro, nemmeno il pi� bravo, pu� sostituirsi all'investigatore nell'analisi dei dati ricavati da un sistema informatico di qualsiasi natura: la mole delle informazioni da visionare � tale e tanta che ci vorrebbero anni di lavoro-uomo per considerarle tutte. La bravura dell'investigatore, in qualsiasi tipo di indagine, � proprio quella di andare a scovare gli elementi utili tra le pieghe delle informazioni che � riuscito a raccogliere: nel caso delle indagini forensi, si tratta di analizzare gigabyte e gigabyte di dati, visto che ormai il prezzo degli hard disk � sceso a tal punto che un disco da un terabyte (1000 gigabyte) � ormai il taglio minimo per un computer desktop.
	\subsection{Fonti normative}
		Con la legge n. 48 del 18 marzo 2008 � stata ratificata la Convenzione del Consiglio d'Europa sulla criminalit� informatica, risalente al novembre 2001. La legge si occupa in generale di criminalit� informatica, e introduce significative modifiche al codice penale in tale ambito. Oltre a queste, tuttavia, sono state inserite anche novit� al Codice di Procedura Penale che risultano di particolare interesse per l'informatica forense.
		\begin{description}
			\item[Art. 244 c.p.p.] Casi e forme delle ispezioni
			\begin{enumerate}
				\item L'ispezione delle persone, dei luoghi e delle cose � disposta con decreto motivato quando occorre accertare le tracce e gli altri effetti materiali del reato.
				\item Se il reato non ha lasciato tracce o effetti materiali, o se questi sono scomparsi o sono stati cancellati o dispersi, alterati o rimossi, l'autorit� giudiziaria descrive lo stato attuale e, in quanto possibile, verifica quello preesistente, curando anche di individuare modo, tempo e cause delle eventuali modificazioni. L'autorit� giudiziaria pu� disporre rilievi segnaletici, descrittivi e fotografici e ogni altra operazione tecnica [359], \textbf{anche in relazione a sistemi informatici o telematici, adottando misure tecniche dirette ad assicurare la conservazione dei dati originali e ad impedirne l'alterazione.}
			\end{enumerate}
			\item[Art. 247 c.p.p.] Casi e forme delle perquisizioni
			\begin{labeling}{1-bis.}
				\item[\textellipsis]
				\item[1-bis.] Quando vi � fondato motivo di ritenere che dati, informazioni, programmi informatici o tracce comunque pertinenti al reato si trovino in un sistema informatico o telematico, ancorch� protetto da misure di sicurezza, ne � disposta la perquisizione, adottando misure tecniche dirette ad assicurare la conservazione dei dati originali e ad impedirne l?alterazione.
				\item[\textellipsis]
			\end{labeling}
			
		\end{description}
		
		
		
		
\end{document}